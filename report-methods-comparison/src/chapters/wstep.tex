\section{Wstęp}

Niniejsza praca jest częścią projektu mającego na celu identyfikację zespołu QRS w sygnale z elektrokardiografu. Sygnał EKG zaraz po zebraniu danych jest zaszumiony i wymaga wygładzenia oraz usunięcia zakłóceń. Największą przeszkodą w analizie sygnału są zakłócenia niskiej częstotliwości, które mogą pochodzić od ruchów pacjenta (ang. \textit{motion artifacts}), jego oddychania lub zmiany rezystancji połączenia elektrody ze skórą. Zakłócenia wysokiej częstotliwości, które mogą utrudniać interpretację sygnału to szumy pochodzące od elektromiogramu i napięcia zasilającego 50Hz.

Jednym ze sposobów wstępnego przetwarzania sygnału EKG zanim przystąpi się do jego analizy jest filtr Savitzky-Golay.

W naszej części projektu przygotowujemy prototyp algorytmu z użyciem języka Python, a następnie implementację w języku C++. Dane wykorzystane podczas testów to sygnały z bazy MIT-BIH o numerach 100, 101 oraz 117.
Użyte w projekcie narzędzia przedstawia tabela \ref{tab:tools}.

\begin{table}[!htb]
  \centering
  \begin{tabular}{|c|c|c|c|}
  \hline
  Część projektu & Język programowania & Biblioteki & Obrazowanie \\
  \hline
  Prototyp & Python 3.5 & NumPy & Matplotlib \\
  \hline
  Implementacja & C++ & Eigen & gnuplot \\
  \hline
  \end{tabular}
  \caption{Zestawienie języków programowania i modułów}
  \label{tab:tools}
\end{table}
