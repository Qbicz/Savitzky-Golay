\section{Wstęp}

W ramach przedmiotu \textit{Elektroniczne systemy diagnostyki medycznej i terapii} zadaniem czterech grup dwuosobowych było przygotowanie algorytmów wstępnego przetwarzania sygnału EKG. Niniejsze sprawozdanie porównuje metody filtracji, badając jakie zakłócenia można z ich pomocą usunąć i jakiego czasu wymaga ich wykonanie.

Praca przebiegała w następujący sposób: w pierwszej fazie przygotowano prototypy z użyciem języków skryptowych. Te wersje programów użyto do dobierania parametrów i badania zachowania algorytmów. Następnie grupy zrealizowały implementacje w języku C++ z użyciem biblioteki Eigen, w celu przyspieszenia programów i zmniejszenia ich zapotrzebowania na zasoby. Następnie wyniki zostały porównane, a algorytmy poddano ocenie której wyniki znalazły się w niniejszym opracowaniu.

Metody filtracji, które zostały zrealizowane, to metoda nielokalnych średnich (NLM), empiryczny rozkład modalny (EMD), filtracja z użyciem falek Wienera (WWF) i filtr Savitzky-Golay (SG). Zestawienie nazw, skrótów które będą używane w sprawozdaniu oraz języków prototypowania przedstawia tabela \ref{tab:skroty}.

\begin{table}[!htb]
  \centering
  \begin{tabular}{|c|c|c|c|}
  \hline
  Metoda filtracji & Skrót & Język prototypowania \\
  \hline
  \textbf{Nonlocal Means} & NLM & Matlab \\
  \hline
  \textbf{Empirical Mode Decomposition} & EMD & Matlab \\
  \hline
  \textbf{Wiener Wavelet Filter} & WWF & Matlab \\
  \hline
  \textbf{Savitzky-Golay} & SG & Python + SciPy \\
  \hline
  \end{tabular}
  \caption{Zestawienie zrealizowanych algorytmów}
  \label{tab:skroty}
\end{table}
