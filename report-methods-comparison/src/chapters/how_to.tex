\section{Instrukcje} \label{sec:how_to}

%Dodatek: jak zainstalować, jak uruchomić, jak testować

Dodatek ten opisuje sposoby kompilacji i uruchomienia przygotowanej aplikacji.

\subsection{Uruchomienie prototypu}

Uruchomienie prototypu wymaga zainstalowania środowiska Python.
Popularną dystrybucją zawierającą niezbędne moduły jest pakiet Anaconda.

\noindent Uruchomienie prototypu z linii komend:
\begin{verbatim}
  prototype$ ./savitzky_design.py
\end{verbatim}


\subsection{Kompilacja aplikacji}

Do budowania aplikacji wykorzystano kompilator \textbf{gcc} w wersji 5.4.0.\\
Flagi kompilacji: \textit{-Ofast -Wall -std=c++11}.
W celu ułatwienia budowy aplikacji w projekcie wykorzystano framework \textbf{cmake}. Instrukcję pobrania instalacji można znaleźć
na stronie twórców narzędzia: https://cmake.org/ .\\
\noindent Procedura kompilacji z użyciem cmake:
\begin{verbatim}
  implem/build$ cmake ../
  implem/build$ make
\end{verbatim}


Alternatywą do systemu budowania cmake może być zwykły system oparty o pliki Makefile lub nawet bezpośrednie użycie kompilatora wraz ze wspomnianymi flagami kompilacji.\\

\subsection{Uruchomienie aplikacji}

\noindent Uruchomienie aplikacji odbywa się za pośrednictwem linii poleceń:
\begin{verbatim}
  implem/build$ ./SavGol.o
\end{verbatim}


\noindent Po uruchomieniu programu bez żadnych argumentów otrzymamy wiadomość zwrotną wraz z poprawnym użyciem programu i opisem poszczególnych parametrów:
\begin{verbatim}
  implem/build$ ./SavGol.o
  Usage:
  ./SavGol.o -i file -o file [-n N] [-m M] [plot]
  Options:
   --input, -i file  : name of the input file
   --output, -o file : name of the output file
   --plot, -p        : plot filtered signal at the end; default: disabled
   -n N              : half of the filtering range; default: 2
   -m M              : half of the filtering range; default: 5
\end{verbatim}



\noindent W ramach projektu przygotowano skrypt, który zbuduje i uruchomi aplikację:
\begin{verbatim}
  implem$ ./runSavitzkyGolay
\end{verbatim}

W celu zmiany parametrów uruchomienia najłatwiej poddać skrypt modyfikacji i uruchomić go ponownie.\\

Do poprawnego wyświetlania wykresów niezbędne jest środowisko \textbf{gnuplot}. Środowisko to jest z założenia międzyplatformowe, więc powinno działać na większości systemów operacyjnych. Instrukcja jego instalacji znajduje sie pod adresem http://www.gnuplot.info/ .\\




Aktualny kod projektu dostępny w repozytorium https://github.com/Qbicz/Savitzky-Golay
