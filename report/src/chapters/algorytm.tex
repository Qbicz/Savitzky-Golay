\section{Algorytm}

Filtr Savitzky-Golay pozwala wygładzić cyfrowy sygnał. Jego działanie opiera się na lokalnym przybliżeniu próbek sygnału wielomianami niskiego rzędu \cite{whatissg}.

Mając dany sygnał x[n], poszukujemy wielomianu
\begin{equation}
p(n) = \sum\limits_{k=0}^N a_k n^k
\end{equation}
który w otoczeniu $2M+1$ punktów minimalizuje kwadrat błędu aproksymacji
\begin{equation}
\epsilon_N = \sum\limits_{n=-M}^M (p(n) - x[n])^2
\end{equation}

Wtedy dla próbki centralnej pośród $2M+1$ punktów (ma ona indeks $n=0$) wartość odfiltrowanego sygnału przyjmuje
\begin{equation}
y[0] = p(0) = a_0
\end{equation}

W praktyce filtr Savitzky-Golay realizuje swoje zadanie obliczając splot 2M+1 aktualnie branych pod uwagę punktów z wielomianową aproksymacją ciągu z jednostkowym impulsem w środku sekwencji.
\begin{equation}
y[n] = \sum\limits_{m=n-M}^{n+M} h[n-m] x[m]
\end{equation}
gdzie $h$ jest wynikiem interpolacji wielomianem stopnia $N$ sekwencji z jednostkowym impulsem, np. $[0, 0, 0, 1, 0, 0, 0]$ dla $M=3$.


\subsection{Przyjęte parametry}
Opracowanego w pracy filtru można użyć w filtracji sygnału EKG na dwa sposoby: do odfiltrowania zakłóceń wysokiej częstotliwości oraz do wyznaczenia izolinii reprezentującej na przykład ruchy związane z oddychaniem pacjenta. Mając wyznaczoną izolinię można potem odjąć ją od sygnału wejściowego otrzymując poprawny sygnał EKG. Tak rozbieżne cele wymagają użycia różnych rozmiarów okna filtracji. Parametry przyjęte w pracy przedstawia poniższa Tablica \ref{tab:params}

\begin{table}[!htb]
  \centering
  \begin{tabular}{|c|c|c|c|}
  \hline 
  Cel / Parametr  & N & M \\  
  \hline 
  Wyznaczenie izolinii & 2 & 900 \\
  \hline
  Odfiltrowanie zakłóceń wysokiej częstotliwości & 2 & 5 \\
  \hline
\end{tabular} 
\caption{Parametry filtru}
\label{tab:params}
\end{table}
